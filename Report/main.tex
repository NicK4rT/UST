\documentclass[10pt,a4]{article}
\usepackage[utf8]{inputenc}
\usepackage[top=20mm, left=20mm, bottom=20mm, right=20mm]{geometry}
\usepackage{svg}
\usepackage{graphicx}
\usepackage[nosumlimits]{amsmath}

\usepackage[
    backend=biber, 
    natbib=true,
    style=numeric,
    sorting=none
]{biblatex}

\addbibresource{references.bib}

\title{\vspace{.0cm}Urban Systems and Transportation \\ Final Project \vspace{5cm}\\ \textbf{Regional Transfers to Fund Resilient Local Communities}\vspace{1 cm}}
\author{Jan-Philip Erdmann, Samuel Luther, Nils Renziehausen, \\Ane Nielsen Solberg, Nicolas Y.C. Triebold\vspace{1cm}}
\date{December 2022}

\begin{document}
\pagenumbering{gobble}
\maketitle
\newpage
\pagenumbering{arabic}

%\section{Table of Contents}

%\newpage
\begin{abstract}
Abstract of report
\end{abstract}
\section{Introduction}
As part of the autumn semester 2022 course \textit{Urban Systems and Transportation}\cite{course} at ETH Zurich, we have executed research looking into the effects of climate change related exposures on regions of The United Kingdom (UK), defined and calculated local and external economical costs, and build a model which simulates the reaction of individuals and firms to transfers. The model explores different transfer strategies and are evaluated by the metrics and welfare function. \\\\
In accordance with the given assignment specifications for the final project, we have decided to focus on the United Kingdom (UK) and build an economical cost model based on risk factors for climate change exposures, as well as transportation and infrastructure assets. The report compares geographical locations at the regional level, as seen in the map and list of statistical regions in Figure \ref{dg:ITL Uk}.\\
\begin{figure}[h]
  \centering
  \begin{minipage}{.5\textwidth}
      \centering
      \resizebox{0.8\textwidth}{!}{\includesvg[width=200pt]{NUTS_1_statistical_regions_of_the_United_Kingdom_map.svg}}
      \caption{ITL - statistical regions in the UK \cite{zotero-176}}
      \label{dg:ITL Uk}
  \end{minipage}%
  \begin{minipage}{.5\textwidth}
    \vspace{-2cm}
    \begin{itemize}
            \item[] \hspace{-.5cm}\textbf{Legend:}
            \item[]
            \item \textbf{C} North East England
            \item \textbf{D} North West England
            \item \textbf{E} Yorkshire and the Humber
            \item \textbf{F} East Midlands
            \item \textbf{G} West Midlands
            \item \textbf{H} East of England
            \item \textbf{I} London
            \item \textbf{J} South East England
            \item \textbf{K} South West England
            \item \textbf{L} Wales
            \item \textbf{M} Scotland
            \item \textbf{N} Northern Ireland
        \end{itemize}
    \end{minipage}
\end{figure}

\section{Data collection}

\subsection{Risk data}
Risk data was gathered in order to define the exposure of climate changes occurring in relation to different sets of assets. 
\subsubsection{Temperature increase}

\subsubsection{Rainfall}
United Kingdom Climate Projections User Interface (UKCP UI) provides tools and data of the latest updates for climate projections, enabling users to follow future climate trends in the UK. The service provide regional, global, probabilistic, marine and derived projections. We used their data sets for gathering data of the standard deviation of $mm$ rainfall in each region of the UK. \cite{}[cite! I could not access the web page can you put the link?]
\subsubsection{Flood risk}
The gov.uk official statistics website offers data on floods and flood risk and it provides the possibility for reviewing the long-term flood risk exposure for specific addresses. However, the underlying data set is not made public. Hence, we used a sampling of $N$ random addresses in the different regions to estimate the regional flood risk. \cite{where}
\subsubsection{$CO_2$ emissions}
Regional $CO_2$ emissions data from 2005 to 2019 could be directly sourced and downloaded from the official gov.uk statistics website. \cite{uk}
\subsubsection{Fire risk}
Regional fire risk data was difficult to detect and the only usable data we could access was charts denoting the fire number and burnt area per year for the countries of England, North Ireland, Scotland and Wales. These were manually transcribed. \cite{countryprofile}
\subsection{Asset data}
Asset data were gathered to define a set of available resources with economical value in the UK regions as well as population data. The value of the resources available and risks related to climate change on the specific asset will define what costs an exposure will inflict. 
\subsubsection{GDP}
GDP was used as a measure for the size of the regional economies.  
\subsubsection{Motorway infrastructure}
\subsubsection{Port infrastructure}
\subsubsection{Energy generation}
\subsubsection{Agricultural land use}
\subsubsection{Population}
\subsubsection{Area}
\subsubsection{Wage - Income}
\subsubsection{Regional productivity}

\section{Monetary risk estimation}
\section{Model}

The model uses the following structure to estimate the welfare after transfer payments and reallocation of people. Index $i$ denotes the ITL region.
The model runs multiple times (t = 10). Each run represents one year. The reason is, a change in population though reallocation of people due to a change in attractiveness with the transfer payments the properties of the region are changing. A changing population has an affect on risk per capita formula \ref{risk per capita} and population density formula \ref{population density} with consequences to all following equations. At each consecutive run the newly crated population allocation in the UK is used to calculate the resulting welfare of the transfer schedule.
\newline
\begin{center}
\textbf{\textit{For t = 0,10:}}\vspace{.5cm}

The risk per capita is calculated by formula \ref{risk per capita}:
\begin{equation}
    rpc_{i,t} = \frac{r_{total,i}}{population_{i,t-1}} 
    \label{risk per capita}
\end{equation}

\begin{center}
    $\downarrow$
\end{center}
The population density is provided by formula \ref{population density}:
\begin{equation}
    pd_{i,t} = \frac{population_{i,t-1}}{area_i}
    \label{population density}
\end{equation}

\begin{center}
    $\downarrow$
\end{center}
The following correlation formula \ref{wage pop density} of wages $w_{i,t}$ and population density $pd_{i,t}$ is used to estimate the wage $w_{i,t}$ in each region per year:
\begin{equation}
    w_{i,t} = (0,0359*pd_{i,t}+598,74) * 52
    \label{wage pop density}
\end{equation}

\begin{center}
    $\downarrow$
\end{center}
Following correlation formula \ref{housing price} of real estate prices $h_i$ to population density $pd_{i,t}$, wages $w_{i,t}$, and fininancing rate of over 20 years is found to estimate the yearly amount spend by residents on housing $h_{i,t}$ in each region:
\begin{equation}
%unkown at the moment NILS has to tell me
    h_{i,t} = \frac{(0.0014*pd_{i,t}+7.0373) * w_{i,t}}{20}
    \label{housing price}
\end{equation}

\begin{center}
    $\downarrow$
\end{center}
A transfer payment scheme is used to reallocate money from one region to another according to an underlying principle. All transfer schedules have in common that a tax between 0\% and 100\% is applied. Every rate is tired out. E.g. transfer schedule 1 formula \ref{transfer schedule 1} where the transferred amount depends only on the regional per capita risk level. See further down for other transfer schedules:
\newline
\newline
\textbf{\textit{For $tax$ = 0\% to 100\%:}}
\begin{equation}
    t_{i,t,tax} = tax * \sum_i w_{i,t} * \frac{rpc_{i,t}}{\sum_i rpc_{i,t}} - tax * w_{i,t}
    \label{transfer schedule 1}
\end{equation}

\begin{center}
    $\downarrow$
\end{center}
The attractiveness of a given regions is provided by formula \ref{attractiveness region}: 
\begin{equation}
    v_{i,t,tax} = \frac{b_i*(w_{i,t}+t_{i,t,tax}-h_{i,t})}{rpc_{i,t}}
    \label{attractiveness region}
\end{equation}

\begin{center}
    $\downarrow$
\end{center}
Formula \ref{attractiveness region comparison} gives a comparison of the attractiveness of regions and makes them thereby comparable:
\begin{equation}
    \Pi_{i,t,tax} = \frac{v_{i,t,tax}}{\sum_i v_{i,t,tax}}
    \label{attractiveness region comparison}
\end{equation}

\begin{center}
    $\downarrow$
\end{center}
The attractiveness of a region $\Pi_{i,t,tax}$ leads to a population allocation accordingly with formula \ref{population share}. $H_{total}$ corresponds to the total population of the UK:
\begin{equation}
    H_{i,t,tax} = \Pi_{i,t,tax} * H_{total}
    \label{population share}
\end{equation}

\begin{center}
    $\downarrow$
\end{center}
The productivity of the region is given by formula \ref{productivity} by scaling the population with regional productive factor $pf_i$:
\begin{equation}
%pf_productivity facotr
    A_{i,t,tax} = H_{i,t,tax} * pf_i
    \label{productivity}
\end{equation}

\begin{center}
    $\downarrow$
\end{center}
Total welfare $W$ is the sum of all regional productivities given by formula \ref{total welfare}:
\begin{equation}
    W_{t,tax} = \sum_i A_{i,t,tax}
    \label{total welfare}
\end{equation}
\end{center}
\textbf{\textit{Next $tax$}}
\newline
\textbf{\textit{Next t}}

\subsection{Determination of parameter $b_i$}
To implement the model and make it run following steps were required.
First, the unknown factor $b_i$ in formula \ref{attractiveness region} has to be estimated. We have the assumption that currently the risk is considered by people living in each region, therefore the population given by the model through formula \ref{population share} has to be equal to the actual observed regional population in the UK. Using the formulas \ref{risk per capita} to \ref{population share} in MS-Excel with a Solver with the aim to fulfill formula \ref{solver equation} under variation of $b_i$ the $b_i$ for each region is estimated.
\begin{equation}
    H_i(b_i) \stackrel{!}{=}  H_{actual,i}
    \label{solver equation}
\end{equation}

\subsection{Transfer schedule 1}
The first transfer schedule is already mentioned previously. See formula \ref{transfer schedule 1} or below.
\begin{equation}
    t_{i,t,tax} = tax * \sum_i w_{i,t} * \frac{rpc_{i,t}}{\sum_i rpc_{i,t}} - tax * w_{i,t}
    \label{transfer schedule 1 again}
\end{equation}
The motivation for this transfer schedule is the idea that we just look at the exposed risk for each region and allocate resources accordingly. Regions with low per capita risk "subsidise" regions with a high risk per capita.

\subsection{Transfer schedule 2}

\begin{equation}
    t_{i,t,tax} = tax * \sum_i w_{i,t} * \frac{\frac{rpc_{i,t}}{\sum_i rpc_{i,t}} + \frac{pd_{i,t}}{\sum_i pd_{i,t}}}{2} - tax * w_{i,t}
    \label{transfer schedule 2}
\end{equation}
Formula \ref{transfer schedule 2} gives the second transfer schedule. The underlying idea is that highly populated areas should be prioritized over sparsely populated areas since densely populated area can be saved on a lower per capita effort and a relocation few people is easier than of many.

\subsection{Transfer schedule 3}
The third transfer schedule is a combination of the first two. \textbf{GIVE EXPLANATION, better formula maybe}

\begin{equation}
    t_{i,t,tax} = \frac{tax * \sum_i w_{i,t} * \frac{rpc_{i,t}}{\sum_i rpc_{i,t}} - tax * w_{i,t} + tax * \sum_i w_{i,t} * \frac{\frac{rpc_{i,t}}{\sum_i rpc_{i,t}} + \frac{pd_{i,t}}{\sum_i pd_{i,t}}}{2} - tax * w_{i,t}}{2} 
    \label{transfer schedule 3}
\end{equation}


\section{Results/Evaluation}
%whatever sounds better to you

\section{Conclusion}

\printbibliography
\end{document}
